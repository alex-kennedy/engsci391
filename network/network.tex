% !TEX TS-program = xelatex
% !TEX encoding = UTF-8 Unicode

\documentclass[11pt]{article}
\usepackage[margin=2.54cm]{geometry}
\usepackage[usenames, dvipsnames]{color}
\usepackage{graphicx}
\usepackage{tikz}
\usepackage{wrapfig} 
\usepackage{booktabs}
\usepackage{amsmath}
\usepackage{paralist} 
\usepackage{verbatim}
\usepackage{subfig}
\usepackage[parfill]{parskip}
\usepackage{amssymb}
\usepackage{cancel}
\usepackage{fancyhdr}
\usepackage{sectsty}
\usepackage{minted}
\usepackage{multirow}
% \usepackage[table,xcdraw]{xcolor}

% Table of Contents
\usepackage[nottoc,notlof,notlot]{tocbibind}
\usepackage[titles,subfigure]{tocloft}
\renewcommand{\cftsecfont}{\rmfamily\mdseries\upshape}
\renewcommand{\cftsecpagefont}{\rmfamily\mdseries\upshape}

% Headers and footers
\pagestyle{fancy}
\renewcommand{\headrulewidth}{0pt}
\lhead{}\chead{}\rhead{}
\lfoot{}\cfoot{\thepage}\rfoot{}

% Fonts


\geometry{a4paper}
\graphicspath{ {images/} }

% Course colours
\definecolor{engsci314}{RGB}{32, 131, 197}
\definecolor{engsci343}{RGB}{143, 62, 151}
\definecolor{engsci391}{RGB}{231, 31, 99}
\definecolor{scigen201}{RGB}{67, 160, 71}
\definecolor{stats210}{RGB}{239, 65, 55}

\begin{document}
\begin{titlepage}
	\newcommand{\HRule}{\rule{\linewidth}{0.5mm}} % Defines a new command for horizontal lines, change thickness here

	\center

	%------------------------------------------------
	%	Headings
	%------------------------------------------------

	\textsc{\LARGE }\\[1.5cm]
	\textsc{\Large 2018}\\[0.5cm]
	\textsc{\large Semester 1}\\[1cm]

	{\color{engsci391}{
		\begin{tabular*}{\textwidth}{c @{\extracolsep{\fill}} cc}
		\specialrule{0.2em}{.05em}{.05em}
		\end{tabular*}
		}
	}\\[0.4cm]

	{\huge\bfseries Computing Assingment Part II}\\[0.6cm]
	{\large ENGSCI 391}\\[1cm]

	{\color{engsci391}{
		\begin{tabular*}{\textwidth}{c @{\extracolsep{\fill}} cc}
		\specialrule{1em}{.05em}{.05em}
		\end{tabular*}
		}
	}\\[1cm]

	{\large\textit{Author}}\\
	Alex \textsc{Kennedy\\[1cm]}
	aken327\\
	460783474

	\vfill\vfill\vfill
	{\large\today}
	\vfill

\end{titlepage}

\section*{MATLAB Setup}

From the assignment brief, the following vectors were defined. They represent $\phi$, $G$, $K$, and $d$ respectively. 

Complete code lies at the end of this document.

\inputminted[firstline=1, lastline=4]{matlab}{main.m}

\section*{Relaxed Constraints on Line and Generator Capacities}
For this question, the problem was solved with each generator's capacity being larger than the total demand.

As expected, the optimal solution consisted of all supply coming from the (tied) cheapest generator (CLY), as there was no reason for power to be generated elsewhere more expensively and there is no cost for propagation across the whole network. 

\section*{Relaxed Constraints on Line Capacities}
With line capacities relaxed, the optimal solution is produced by fully utilising each generator in order of the cheapest to most expensive, until all supply is met. 

\section*{Treatment of Line Capacities}

\begin{table}[h]
	\centering
	\begin{tabular}{@{}lr@{}}
	\toprule
	Generator & Utilisation (MW) \\ \midrule
	HLY       & 1,000             \\
	E3P       & 52               \\
	OTA       & 400              \\
	MRP       & 1,085             \\
	SFD       & 336              \\
	TKU       & 0                \\
	WTK       & 1,270             \\
	CLY       & 800              \\
	MAN       & 850              \\ \bottomrule
	\end{tabular}
	\caption{Generation by each generator}
\end{table}




\newpage
\section*{Complete MATLAB Code}
A complete listing of the MATLAB code follows. 

\inputminted{matlab}{main.m}


\end{document}
