% !TEX TS-program = xelatex
% !TEX encoding = UTF-8 Unicode

\documentclass[11pt]{article}
\usepackage[margin=2.54cm]{geometry}
\usepackage[usenames, dvipsnames]{color}
\usepackage{graphicx}
% \usepackage{tikz}
\usepackage{wrapfig} 
\usepackage{booktabs}
\usepackage{amsmath}
\usepackage{paralist} 
\usepackage{verbatim}
\usepackage{subfig}
\usepackage[parfill]{parskip}
\usepackage{amssymb}
\usepackage{cancel}
\usepackage{fancyhdr}
\usepackage{sectsty}
\usepackage{minted}
\usepackage{multirow}

% Table of Contents
\usepackage[nottoc,notlof,notlot]{tocbibind}
% \renewcommand{\cftsecfont}{\rmfamily\mdseries\upshape}
% \renewcommand{\cftsecpagefont}{\rmfamily\mdseries\upshape}

% Headers and footers
\pagestyle{fancy}
\renewcommand{\headrulewidth}{0pt}
\lhead{Alex Kennedy}\chead{}\rhead{ENGSCI 391 Assignment 2}
\lfoot{}\cfoot{\thepage}\rfoot{}

% Fonts


\geometry{a4paper}
\graphicspath{ {images/} }

% Course colours
\definecolor{engsci314}{RGB}{32, 131, 197}
\definecolor{engsci343}{RGB}{143, 62, 151}
\definecolor{engsci391}{RGB}{231, 31, 99}
\definecolor{scigen201}{RGB}{67, 160, 71}
\definecolor{stats210}{RGB}{239, 65, 55}

\begin{document}
\begin{titlepage}
	\newcommand{\HRule}{\rule{\linewidth}{0.5mm}} % Defines a new command for horizontal lines, change thickness here

	\center

	%------------------------------------------------
	%	Headings
	%------------------------------------------------

	\textsc{\LARGE }\\[1.5cm]
	\textsc{\Large 2018}\\[0.5cm]
	\textsc{\large Semester 1}\\[1cm]

	{\color{engsci391}{
		\begin{tabular*}{\textwidth}{c @{\extracolsep{\fill}} cc}
		\specialrule{0.2em}{.05em}{.05em}
		\end{tabular*}
		}
	}\\[0.4cm]

	{\huge\bfseries Computing Assignment Part II}\\[0.6cm]
	{\large ENGSCI 391}\\[1cm]

	{\color{engsci391}{
		\begin{tabular*}{\textwidth}{c @{\extracolsep{\fill}} cc}
		\specialrule{1em}{.05em}{.05em}
		\end{tabular*}
		}
	}\\[1cm]

	{\large\textit{Author}}\\
	Alex \textsc{Kennedy\\[1cm]}
	aken327\\
	460783474

	\vfill\vfill\vfill
	{\large\today}
	\vfill

\end{titlepage}

\section{MATLAB Set-up}

From the assignment brief, the following vectors were defined. They represent $\phi$, $G$, $K$, and $d$ respectively. 

\inputminted[firstline=5, lastline=8]{matlab}{main.m}

Complete code lies in Appendix A.

\section{Relaxed Constraints on Line and Generator Capacities}
As expected, the optimal solution consists of all supply coming from the (tied) cheapest generator (CLY at 5,793MW), as there is no reason for power to be generated elsewhere more expensively and there is no cost for, or limit to, propagation across the whole network. 

The total cost of this set up is \$173,790/hr.

\section{Relaxed Constraints on Line Capacities}
With line capacities relaxed, the optimal solution is produced by fully utilising each generator in order of the cheapest to most expensive, until all supply is met. 

The total cost of this set up is \$233,450/hour.  

\section{Treatment of Line Capacities}

The total cost when accounting line capacities is \$241,150/hr. The utilisations are obtained from the optimal $x$ and energy price at each node is obtained from the duals, $\pi$.

\begin{table}[h]
	\begin{minipage}{0.5\linewidth}
		\centering
		\begin{tabular}{@{}lr@{}}
		\toprule
		Generator & Utilisation (MW) \\ \midrule
		HLY       & 1,000             \\
		E3P       & 52               \\
		OTA       & 400              \\
		MRP       & 1,085             \\
		SFD       & 336              \\
		TKU       & 0                \\
		WTK       & 1,270             \\
		CLY       & 800              \\
		MAN       & 850              \\ \bottomrule
		\end{tabular}
		\caption{Generation by each generator}
	\end{minipage}
	~
	\begin{minipage}{0.5\linewidth}
		\centering
		\begin{tabular}{@{}lr@{}}
		\toprule
		Link & Utilisation (MW) \\ \midrule
		A - H       & \color{red}{$-500$}        \\
		H - NP      & \color{red}{$-276$}        \\
		H - N       & \color{ForestGreen}{$139$} \\
		NP - W      & $0$                        \\
		N - W       & \color{red}{$-145$}        \\
		W - B       & \color{red}{$-1000$}       \\
		B - C       & \color{ForestGreen}{$278$} \\
		B - D       & \color{red}{$-8$}          \\
		C - D       & \color{red}{$-800$}        \\
		T - D       & \color{ForestGreen}{$233$} \\
		M - T       & \color{ForestGreen}{$850$} \\ \bottomrule
		\end{tabular}
		\caption{Intercity transfer. Negative values represent flows against the direction specified.}
	\end{minipage}
\end{table}

\begin{table}[h]
	\centering
	\begin{tabular}{@{}lr@{}}
	\toprule
	City & Extra Cost (\$/MW) \\ \midrule
	A       & 60              \\
	H       & 55              \\
	NP      & 55              \\
	N       & 55              \\
	W       & 55              \\
	B       & 40              \\
	C       & 40              \\
	D       & 40              \\
	T       & 40              \\
	M       & 30              \\ \bottomrule
	\end{tabular}
	\caption{Extra cost per MW of power in each city (the vector of duals)}
\end{table}

\section{Kirchhoff's Laws}
Perhaps surprisingly, generator utilisation and duals were the same as the previous section (Tables 1, 3), as was the optimal value (\$241,150/hr) when enforcing Kirchhoff's laws. Only distribution about the network has altered.

\begin{table}[h]
	\centering
	\begin{tabular}{@{}lr@{}}
	\toprule
	Link & Utilisation (MW) \\ \midrule 
	A - H       & \color{red}{$-500$}           \\
	H - NP      & \color{red}{$-208.5$}         \\
	H - N       & \color{ForestGreen}{$71.5$}   \\
	NP - W      & \color{ForestGreen}{$67.5$}   \\
	N - W       & \color{red}{$-212.5$}         \\
	W - B       & \color{red}{$-1000$}          \\
	B - C       & \color{ForestGreen}{$292.2$}  \\
	B - D       & \color{red}{$-22.2$}          \\
	C - D       & \color{red}{$-785.8$}         \\
	T - D       & \color{ForestGreen}{$233$}    \\
	M - T       & \color{ForestGreen}{$850$}    \\ \bottomrule
	\end{tabular}
	\caption{Intercity transfer, accounting for Kirchhoff's Laws. Negative values represent flows against the direction specified.}
\end{table}

\section{Line Losses in the HVDC Cook Straight Line}
To model losses in the Blenheim - Wellington HVDC line, the existing arcs were first disconnected. Then, an arc with capacity 500MW was connected (in both directions) where only 95\% of the electricity exiting the relevant node arrives at the destination. A further arc, also capped at 500MW, in both directions, was added as before, but with only 85\% energy arriving. This means a maximum of 1000MW can be transferred. The RSM will naturally favour the arc with less loss, simulating the effect described in the assignment. 

The cost of this set up is \$246,895/hr. 

\begin{table}[h]
	\begin{minipage}{0.5\linewidth}
		\centering
		\begin{tabular}{@{}lr@{}}
		\toprule
		Generator & Utilisation (MW)  \\ \midrule
		HLY       & 1,000             \\
		E3P       & 101               \\
		OTA       & 400               \\
		MRP       & 1,085             \\
		SFD       & 387               \\
		TKU       & 0                 \\
		WTK       & 1,270             \\
		CLY       & 800               \\
		MAN       & 850               \\ \bottomrule
		\end{tabular}
		\caption{Generation by each generator}
	\end{minipage}
	~
	\begin{minipage}{0.5\linewidth}
		\centering
		\begin{tabular}{@{}lr@{}}
		\toprule
		Link & Utilisation (MW) \\ \midrule
		A - H       & \color{red}{$-451$}          \\ % 1
		H - NP      & \color{red}{$-196.75$}       \\ % 2
		H - N       & \color{ForestGreen}{$108.75$}\\ % 3
		NP - W      & \color{ForestGreen}{$130.25$}\\ % 4
		N - W       & \color{red}{$-175.25$}       \\ % 5
		W - B       & \color{red}{$-1000$}          \\ % 6
		B - C       & \color{ForestGreen}{$292.2$}  \\ % 7 
		B - D       & \color{red}{$-22.2$}          \\ % 8
		C - D       & \color{red}{$-785.8$}         \\ % 9
		T - D       & \color{ForestGreen}{$233$}    \\ % 10
		M - T       & \color{ForestGreen}{$850$}    \\ \bottomrule
		\end{tabular}
		\caption{Intercity transfer. Negative values represent flows against the direction specified.}
	\end{minipage}
\end{table}

\begin{table}[h]
	\centering
	\begin{tabular}{@{}lr@{}}
	\toprule
	City & Extra Cost (\$/MW) \\ \midrule
	A       & 60              \\
	H       & 60              \\
	NP      & 60              \\
	N       & 60              \\
	W       & 60              \\
	B       & 40              \\
	C       & 40              \\
	D       & 40              \\
	T       & 40              \\
	M       & 30              \\ \bottomrule
	\end{tabular}
	\caption{Extra cost per MW of power in each city (the vector of duals)}
\end{table}

\newpage\phantom{hi marker :)}

\newpage
\section*{Appendix A $\quad$ Complete MATLAB Code}
A complete listing of the MATLAB code follows. 

\inputminted{matlab}{main.m}


\end{document}
